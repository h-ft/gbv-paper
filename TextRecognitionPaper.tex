\documentclass[a4paper]{article}
\usepackage[english]{babel}
\usepackage[utf8x]{inputenc}
% package for including graphics with figure-environment
\usepackage{graphicx}
\usepackage{hyperref}
% colors for hyperlinks
% colored borders (false) colored text (true)
\hypersetup{colorlinks=true,citecolor=black,filecolor=black,linkcolor=black,urlcolor=black}

% package for bibliography
\usepackage[authoryear,round]{natbib}
% package for header
\usepackage[automark,headsepline]{scrlayer-scrpage}
\pagestyle{scrheadings}
\ihead[]{Text Detection in Natural Scene Images}
\ohead[]{September XX, 2021}
\cfoot[]{\pagemark} 

\begin{document}
	\title{
	\begin{figure}[!ht]
		% \flushleft
			\includegraphics[width=0.26\textwidth]{img/htwlogo.jpg}
	\end{figure}
	\vspace{1cm}
	\Huge Text Detection in Natural Scene Images \\
	with CRAFT and EAST
	}
	
	\vspace{1cm}
	
	% if you are the only author, you might use the following
	% \author{Name of student}	
	
	% Insert here your name and correct mail address
	\author{\Large \href{mailto:s0566146@htw-berlin.de}{Henry Febrian}
	\vspace{1cm}}
	
	% name of the course and module
	\date{
	\large Module: \textit{Computergrafik und Bildverarbeitung} \\
	\vspace{0.8cm}
	\large Lecturer: Erik Rodner \\
	\vspace{1cm}
	\large September XX, 2021
	}

	\maketitle
	\setlength{\parindent}{0pt}

\vspace{2cm}
\begin{abstract}
To be edited at a later date. 

\end{abstract}
	\newpage
	\tableofcontents
	\newpage
	
\section{Introduction} % (fold)
\label{sec:introduction}
Text is arguably one of the most essential form of communication. According to \cite{LongEtAl}: 
\begin{quotation}
	\emph{``As the written form of human languages, text makes it feasible to reliably and effectively spread or acquire information across time and space. In this sense, text constitutes the cornerstone of human civilization.''} 
	\citep{LongEtAl} 
\end{quotation}
In the modern world, text as a medium of communication is not only consumed by humans but has claimed its place in the world of technology.
However, text detection in natural scene images is proven to be challenging. Compared to detecting text on handwritten materials, the randomness of a natural scene is a big hurdle to overcome.
This paper begins by observing the interference found on natural scenes followed by introducing and giving an overview of CRAFT and EAST.
It then presents an overview on the how text is detected using the two methods, along with an explanation of the evaluation dataset. Subsequently, the performance review of the algorithm will be presented, which is obtained by testing the preferred method on the aforementioned dataset.
% section introduction (end)

\section{Definition of Natural Scene Images} % (fold)
\label{sec:definition}
asdasdssada
% section definition (end)

\section{Challenges of Natural Text Detection} % (fold)
\label{sec:challenges}
Natural text detection is a greater challenge than detecting structured text in documents. \cite{NaturalScene} mentioned some conditions that are found in natural scene which may significiantly impact text detection procedure. They are:
\begin{itemize}
	\item Raw sensor image and sensor noise
	\item Viewing angle
	\item Blur
	\item Lighting
	\item Resolution and Aliasing
\end{itemize}
% section challenges (end)

\section{Overview of EAST} % (fold)
\label{sec:overvieweast}
According to \cite{EastZhouEtAl}, the key component of EAST is a neural network model, which is trained to directly predict the existence of text instances and their geometries from full images \citep{EastZhouEtAl}.
Hence, the abbreviation EAST: Efficient and Accurate Scene Text Detector.
\begin{figure}[h!]
	\centering
	\includegraphics[width=0.5\textwidth]{img/eaststructure.png}
	\caption{Structure of EAST, adopted from~\protect\cite{EastZhouEtAl}}
	\label{fig:east1}
\end{figure}


% section overvieweast (end)

\section{Overview of CRAFT} % (fold)
\label{sec:overviewcraft}
CRAFT stands for Character Region Awareness for Text Detection. According to \cite{CraftBaekEtAl}, CRAFT is a novel text detector which localizes the individual character regions and links the detected characters to a text instance \citep{CraftBaekEtAl}.
\begin{figure}[h!]
	\centering
	\includegraphics[width=0.5\textwidth]{img/craftstructure.png}
	\caption{Structure of Craft, adopted from~\protect\cite{CraftBaekEtAl}}
	\label{fig:craft1}
\end{figure}
% section overviewcraft (end)

\section{Methodology} % (fold)
\label{sec:methodology}
asddssafs
% section methodology (end)

\section{Results and Evaluation} % (fold)
\label{sec:evaluation}
asdasdssada
% section evaluation (end)

\section{Conclusion} % (fold)
\label{sec:conclusion}
Lorem ipsum dolor sit amet, consetetur sadipscing elitr, sed diam nonumy eirmod tempor invidunt ut labore et dolore magna aliquyam erat, sed diam voluptua. At vero eos et accusam et justo duo dolores et ea rebum. Stet clita kasd gubergren, no sea takimata sanctus est Lorem ipsum dolor sit amet. Lorem ipsum dolor sit amet, consetetur sadipscing elitr, sed diam nonumy eirmod tempor invidunt ut labore et dolore magna aliquyam erat, sed diam voluptua. At vero eos et accusam et justo duo dolores et ea rebum. Stet clita kasd gubergren, no sea takimata sanctus est Lorem ipsum dolor sit amet.
% section conclusion (end)

\section{Section about quotations} % (fold)
\label{sec:section_about_quotations}

In this section, an example for a literal quotation is given. 

\begin{quotation}
	\emph{``A persona is a rich picture of an imaginary person who represents your core user group.''}
	\citep{CraftBaekEtAl}
\end{quotation}

Sometimes you might want to make use of the authors name within the text. Before, we used the command \texttt{citep\{\}}, which creates the brackets around author name and year. You can also use the \texttt{cite} command like this: \\

\cite{Dix04} defined the concept of persona as follows: 
\begin{quotation}
	\emph{``A persona is a rich picture of an imaginary person who represents your core user group.''}
	\citep{EastZhouEtAl}
\end{quotation}

You may notice, that this increases the readability of the text. \\

According to APA format\footnote{ American Psychological Association (APA)} there are some rules, when and how to include page numbers, when referring to literature. 

\begin{quotation}
	\emph{``Include page numbers for any citations in the text of your paper that include direct quotations or refer to a specific part of the work you are referencing. Direct quotations must include a page number as part of the citation. The quoted material should be followed by a citation in parentheses that gives the author's name, the year in which the work was published, and the page number from which the quoted material appears.''}
	\citep{Hall}
\end{quotation}

Check out the example and recommendations of \cite{Hall} on \url{http://www.ehow.com/how_5689799_cite-numbers-apa-format.html}. In \LaTeX you can include the pages very easy. For example: \\

\citet[p. 86]{Baddeley:1974ts} stated: 

\begin{quotation}
	\emph{``We hope that our preliminary attempts to begin answering the question will convince the reader, not necessarily that our views are correct, but that the question was and is well worth asking''}
	\citep[p. 86]{Baddeley:1974ts}
\end{quotation}

Note that in the first reference, we used \texttt{citet[]\{\}} in order to have brackets just around year and page number; later we used \texttt{citep[]\{\}}.

% section section_about_quotations (end)


\section{Section about references within the document} % (fold)
\label{sec:section_about_references_within_the_document}

If you want to refer to you own chapters, figures, tables or the like, you can make use of the \texttt{ref\{\}} command, for example:
\begin{itemize}
	\item section~\ref{sec:section_about_quotations} on page \pageref{sec:section_about_quotations}
\end{itemize} 



% section section_about_references_within_the_document (end)

\subsection{Subsection within Foundations} % (fold)
\label{sub:subsection_within_foundations}
Lorem ipsum dolor sit amet, consetetur \cite{NaturalScene} sadipscing elitr, sed diam nonumy eirmod tempor invidunt ut labore et dolore magna aliquyam erat, sed diam voluptua. At vero eos et accusam et justo duo dolores et ea rebum. Stet clita kasd gubergren, no sea takimata sanctus est Lorem ipsum dolor sit amet. Lorem ipsum dolor sit amet, consetetur sadipscing elitr, sed diam nonumy eirmod tempor invidunt ut labore et dolore magna aliquyam erat, sed diam voluptua. At vero eos et accusam et justo duo dolores et ea rebum. Stet clita kasd gubergren, no sea takimata sanctus est Lorem ipsum dolor sit amet.
% subsection subsection_within_foundations (end)

\subsection{Another subsection within Foundations} % (fold)
\label{sub:another_subsection_within_foundations}
Lorem ipsum dolor sit amet, consetetur sadipscing elitr, sed diam nonumy eirmod tempor invidunt ut labore et dolore magna aliquyam erat, sed diam voluptua. At vero eos et accusam et justo duo dolores et ea rebum. Stet clita kasd gubergren, no sea takimata sanctus est Lorem ipsum dolor sit amet. Lorem ipsum dolor sit amet, consetetur sadipscing elitr, sed diam nonumy eirmod tempor invidunt ut labore et dolore magna aliquyam erat, sed diam voluptua. At vero eos et accusam et justo duo dolores et ea rebum. Stet clita kasd gubergren, no sea takimata sanctus est Lorem ipsum dolor sit amet.
% subsection another_subsection_within_foundations (end)


% section foundations (end)

\newpage 

\bibliographystyle{natdin}
	\bibliography{references} % expects file "references.bib"
	\addcontentsline{toc}{section}{References}
\end{document}